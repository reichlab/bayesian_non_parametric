\documentclass{article}
\usepackage{amsmath}
\usepackage{graphicx}
\graphicspath{ {/Users/graham/} }

\begin{document}

%+Title
\title{What's Going On?}
\author{Graham Gibson}
\date{\today}
\maketitle
%-Title

%+Abstract
\begin{abstract}
    What are we doing?????????
\end{abstract}
%-Abstract

%+Contents
\tableofcontents
%-Contents

\section{Models for Disease Process}

\subsection{GP}

Insert model here.  Are there multiple variations on this theme?

\begin{itemize}
\item GP + noise model for $X_t \sim X_{t - 1}, \ldots, X_{t - L}$.  What is variance a function of?
\item Hierarchical structure for adjacent spatial units like clusters of districts in Thailand?
\end{itemize}

\subsection{DP}

Insert models here

\begin{itemize}
\item For $X_t \sim X_{t - 1}, \ldots, X_{t - L}$
\item Possible methods paper, see below
\item Could also end up being used for seasonal disease predictions
\end{itemize}


\subsection{RBF MDN}

Equivalent to mixture of normals for $(X, Y)$, conditioned on $X$

\subsection{Tying together horizons}

\begin{itemize}
\item iterate 1-step-ahead predictions
\item Copulas
\item all remaining horizons in season jointly estimated at once
\item linkages (like copulas, but multivariate marginals) -- reserve for a MHC student to explore??
\end{itemize}

for now, pick something and go with it -- maybe a paper here later?

\section{Models for Reporting Process}

\subsection{Krzysztof's model for dengue in Thailand}

Can be used for either individual level data or case counts.

Insert description here.  Basic idea:

$X_{ij} \sim Poisson(\lambda_i p_{ij})$
$p_{ij} = CDF(j - i | \alpha, \beta)$, where $CDF$ comes from K's survival model.

\subsection{Flu -- Errors survival model}

Reporting delay errors occur according to some model that allows for positive or negative values, and they persist for some number of weeks.

Possibly integrate with seasonal fluctuations in reporting.


\section{Paper Concepts}

In some kind of order from firstish to lastish.

\subsection{Flexible Stick Breaking for Dirichlet Process and Probit Stick Breaking Process Mixture of Regression Models}

In DP and PSBP (is that the right acronym?) models, number of mixture components assigned non-negligible weight is sensitive to interactions between a variety of factors, including the functional form used in the stick breaking process, priors on parameters used in the stick breaking process, and priors on mixture component regression coefficients.  Most commonly (??), in Dirichlet Process mixture of regressions (and possibly PBSP too??) a logistic-linear specification in x is used for stick breaking.  We propose a novel, more flexible functional form for stick breaking using a BNN.  Basically, this allows the model to estimate upper and lower bounds on the weight that should be assigned to a single mixture component.  This lets the model distribute weight across multiple mixture components more easily.  (BUT WE STILL NEED TO EVALUATE RELATIVE TO SOME OTHER OPTIONS/MAKE SURE THIS STORY PANS OUT).

\begin{itemize}
\item Use BNN for breaking process probabilities: more flexible than standard linear form.
\item Compare to 2-3 common alternatives chosen from the following candidates:
  \begin{itemize}
  \item stick breaking prob = logistic(a + bx)
  \item paper with empirical bayes priors for DP
  \item probit stick breaking process?
  \item basic idea: 1-2 "default" implementations and 1-2 "fancy/new" implementations?
  \end{itemize}
\item Simulation Study
\item Apply to infectious disease prediction $h$ steps ahead (not seasonal targets?).  Also compare this to a linear model a la SARIMA?
\item Apply to something else DP mixtures of regressions have been used for previously?
\end{itemize}


\subsection{Hierarchical Non-Parametric Density Estimation}

We would like to borrow information across adjacent spatial units using hierarchical model structures.  We would like to do (conditional) density estimation in a flexible way, using non-parametric or flexible semi-parametric methods.  How can we do that?

\begin{itemize}
\item Hierarchical GP is the most obvious answer
\item Is there a way to do this with a DP flavored method?
\item simulation study
\item Potential application: Groups of adjacent/similar districts in Thailand
\item Can we also estimate clustering/hierarchy based on similarity of time series??
\end{itemize}


\subsection{Integrating Reporting Delays with Prediction in Thailand}

Predictions in Thailand would benefit from integrating a reporting delay model with a disease process model.  In a Bayesian setup, this could use any of the non-parametric methods outlined above and combine with K's reporting delay model.  Note that K's model can work with case count data as briefly outlined above.

\begin{itemize}
\item Province-level prediction in Thailand: 76 provinces
\item at the level of case count data
\item this probably comes after a more basic application of the disease process model that doesn't deal with reporting delays...
\end{itemize}


\subsection{Integrating Reporting Delays with Prediction for Flu}

Evan will be thinking about this, reporting delay stuff different from 




\end{document}
